% !TEX TS-program = pdflatex
% !TEX root = ../tesi.tex

%************************************************
\chapter*{Introduzione}
%\label{chp:Introduzione}
%************************************************
L’inquinamento acustico marino è causato dall’immissione di rumori in quantità eccessiva rispetto a quanto l’ambiente in questione sia in grado di sopportare. Vari studi dimostrano che i rumori provocati dall’uomo siano quelli che provocano più effetti negativi sui vari organismi marini. In particolare, le principali fonti di inquinamento marino sono rappresentate dall’attività di ricerca di combustibili fossili (come gas e petrolio), dai sonar utilizzati per usi militari o civili, dalla costruzione di edifici sulla costa, dall’attività di pesca, dagli impianti eolici e infine dal traffico marittimo (che è in costante aumento, in particolare nel Mar Mediterraneo).
Gli impatti di tutte queste attività possono essere diversi a seconda del tipo di rumore prodotto, di dove esso è localizzato e della sua durata. I rumori provocati dai sonar, ad esempio, possono causare dei danni diretti (come una sordità temporanea o permanente). Tuttavia, anche i rumori più deboli non sono da sottovalutare perché, seppur in modo non immediato e meno evidente, anch’essi possono avere effetti negativi importanti, soprattutto se la loro durata è lunga e/o continua. Inoltre, data l’importanza del suono per gli animali marini e la capacità di propagazione del suono nel loro habitat, l’inquinamento acustico nel mare può avere conseguenze non solo sul singolo individuo, ma sull’intera popolazione marina.



