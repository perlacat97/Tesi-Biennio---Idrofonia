% !TEX TS-program = pdflatex
% !TEX root = ../tesi.tex

%************************************************
\chapter*{Introduzione}
%\label{chp:Introduzione}
%************************************************
L’inquinamento acustico marino è causato dall’immissione di rumori in quantità eccessiva rispetto a quanto l’ambiente in questione sia in grado di sopportare. 
Vari studi dimostrano che i rumori provocati dall’uomo siano quelli che provocano più effetti negativi sui vari organismi marini. 
In particolare, le principali fonti di inquinamento marino sono rappresentate dall’attività di ricerca di combustibili fossili (come gas e petrolio), dai sonar utilizzati per usi militari o civili, dalla costruzione di edifici sulla costa, dall’attività di pesca, dagli impianti eolici e infine dal traffico marittimo (che è in costante aumento, in particolare nel Mar Mediterraneo).
Gli impatti di tutte queste attività possono essere diversi a seconda del tipo di rumore prodotto, di dove esso è localizzato e della sua durata. 
I rumori provocati dai sonar, ad esempio, possono causare dei danni diretti (come una sordità temporanea o permanente). 
Tuttavia, anche i rumori più deboli non sono da sottovalutare perché, seppur in modo non immediato e meno evidente, anch’essi possono avere effetti negativi importanti, soprattutto se la loro durata è lunga e/o continua. 
Inoltre, data l’importanza del suono per gli animali marini e la capacità di propagazione del suono nel loro habitat, l’inquinamento acustico nel mare può avere conseguenze non solo sul singolo individuo, ma sull’intera popolazione marina.
Lo scopo di questa ricerca è correlata all'utilizzo di strumenti scientifici, per capire come il rumore può influenzare la natura e sopratutto l'uomo. 
In questi anni sono state effettuate migliaia di ricerche, sopratutto in mare, per approfondire quanto l'inquinamento acustico possa aver influito e influisce, sull'ambiente marino, e quanto l'uomo, e sopratutto anche le costruzioni e lavori in mare, siano il motivo per cui oggi sia tutto sottovalutato. 
Il rumore, è una componente che molti non hanno quasi mai preso in considerazione. 
L'uomo non lo percepisce quando è sottoacqua, sopratutto perchè si pensa che l'acqua stessa non abbia suono, ma non è assolutamente cosi. 
Il “rumore” sottomarino non è di un unico tipo: ci sono suoni naturali, come quelli prodotti dal vento, dalle tempeste, dalle onde, da una turbolenza, da un sisma, il crepitio del ghiaccio di un iceberg: tutti suoni di origine fisica. 
Poi ci sono i suoni di origine biologica come quelli emessi dagli animali o dovuti ai loro movimenti, ma sono quelli provocati dalle attività umane (imbarcazioni, prospezioni per indagini geologiche, attività militari, ecc.) a mettere in pericolo il benessere degli abitanti del mare. 
I mammiferi marini (in particolare i cetacei odontoceti che utilizzano il biosonar) e molti pesci sono quelli maggiormente colpiti dall’inquinamento acustico perché utilizzano le onde sonore per orientarsi, trovare le prede, localizzare un partner, evitare i predatori e comunicare. 
Molti studi confermano che il rumore contribuisca al declino di popolazione di diverse specie di cetacei o alla loro mancanza di ripresa se in calo demografico. 
I rumori che derivano dall’attività umana sono quelli a bassa frequenza (da 10 a 500 Hz) come la navigazione commerciale e, secondariamente, l’esplorazione sismica. 
Questi suoni subiscono poca attenuazione e consentono quindi la propagazione a lungo raggio. 
Dal 1950 al 2000 il rumore in bassa frequenza è raddoppiato ogni 10 anni ("Inquinamento acustico subacqueo", s.d.). 
Il suono a media frequenza (da 500 Hz a 25 kHz) invece non si propaga su lunghe distanze e contribuisce al rumore ambientale locale e regionale. 
I suoni a media frequenza sono per esempio: onde che si infrangono, spruzzi, formazione e collasso di bolle e precipitazioni atmosferiche. Vari sonar (ad esempio militari e cartografici), così come piccole imbarcazioni, fanno parte invece del rumore causato dall’uomo alle frequenze medie. 
Alle alte frequenze (> 25 kHz), la sorgente del rumore per essere percepita deve essere vicina al ricevitore. 
Ricadono all’interno di queste frequenze: il rumore termico e il risultato del moto browniano delle molecole d'acqua vicino all'idrofono (Hildebrand, 2009).
L’aumento del rumore antropogenico provocato dall’azione dell’uomo dipende dall’aumento del numero di imbarcazioni e della stazza delle navi; anche l'esplorazione petrolifera, del gas e i siti di produzione di queste materie prime, il dragaggio, la costruzione e le attività militari, determinano un aumento impressionante della rumorosità in tutti gli oceani. Negli ultimi dieci anni, le ricerche hanno dimostrato che alcune forme di rumore oceanico possono uccidere,ferire e causare sordità a cetacei e altri mammiferi marini, così come ai pesci. 
In particolare, è stato possibile mettere in relazione una serie di spiaggiamenti e di decessi di mammiferi marini con l'esposizione ai sonar militari. 
Si è infine dimostrato che un rumore intenso produce la riduzione delle capacità riproduttive e una maggiore sensibilità alla malattia.



