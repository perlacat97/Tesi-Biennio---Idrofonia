% !TEX TS-program = pdflatex
% !TEX root = ../tesi.tex

\chapter{Ringraziamenti}
Al termine di questo magnifico percorso mi sembra doveroso ringraziare tutte le persone che mi sono state accanto, in questi lunghi due anni, che hanno raggiunto la loro conclusione.
\@

Ringrazio tutti i docenti del Conservatorio Niccolò Piccinni di Bari, del corso di Musica Elettronica, M° Francesco Abbrescia, M° Pier Carmelo Alfeo, M°Anthony Di Furia, M° Alessio Mastrolillo, grazie per la vostra professionalità e i vostri mille consigli, che spero di aver seguito nel modo più coerente possibile. So per certo che il lavoro fatto insieme, mi abbia aiutato e mi aiuterà sicuramente a capire e comprendere al meglio il mio modo di fare musica. 
\@

Inoltre volevo ringraziare il mio relatore M° Giuseppe Silvi.
\@
 
Non mi sarei mai aspettata che nella conclusione di questo percorso avrei scelto un docente come lei. Ma, nella vita, certe scelte, hanno un loro perché. 
La consapevolezza che il suo sapere, i suoi consigli, le sue idee, hanno contribuito alla realizzazione di questo lavoro e di questo percorso. 
So anche di essere consapevole di non aver colto sin da subito l’importanza della sua conoscenza, ma so anche di non aver mai sottovalutato un giorno di lezione fatto insieme a lei. 
Abbiamo avuto i nostri alti e bassi, ma so per certo che questo traguardo lo devo anche a lei. Alla sua pazienza, al suo darmi mille possibilità in questi anni, e soprattutto grazie perché ad ogni modo porterò con me un bagaglio di conoscenza che le devo riconoscere. 
\@

In ultimo ringrazio il direttore del corso il M° Francesco Scagliola. 
In questi due lunghi anni, ho sempre riportato alla mia memoria, il giorno in cui ho scelto di entrare in questo corso. Ricordo perfettamente che quel giorno, qualcuno mi chiese “Cosa vuoi scegliere, lavorare in azienda otto ore al giorno, o entrare qui e dedicare otto ore al giorno alla musica?”, e io gli risposi “dedicherò ogni ora di ogni giorno alla musica!”. 
Devo ringraziarla, perché da quel giorno, ho davvero dedicato ogni ora e ogni giorno alla musica, grazie alla sua perseveranza e costanza, al suo modo di credere, nonostante il mio precedente percorso, in ogni lavoro e progetto che portassi a termine. 
So di certo, di non essere stata una studentessa modello, ma credo davvero che questo traguardo sia anche merito suo. 
\@

Ringrazio tutti i miei compagni di corso: 
Daniele, Gabriele, Mattoni, tutti i primini, Francesco Vitucci, Gianfranco Pennacchia, grazie perché con voi mi sono sempre sentita a casa, nonostante fossi l’unica in mezzo a voi, con gli occhiali rosa, e la risata troppo squillante per sopportarla, ma fino alla fine siete stati coraggiosi! 
\@

Volevo ringraziare in particolare 
\@

Michele: ho pensato tanto in questi giorni a cosa poter scrivere riguardo la tua persona, e il solo pensiero, mi ha fatto scendere qualche lacrima. 
Dal primo giorno, da quella riunione meet, era impensabile che il nostro rapporto diventasse un’amicizia folle e piena di amore. 
Mi hai aiutato nei momenti davvero più difficili di questo percorso, mi sei stato affianco nel bene o nel male, mi hai sopportato e supportato costantemente, e non mi hai mai fatto sentire fuori luogo, in ogni momento passato assieme, in qualsiasi aula ci trovassimo. 
Mi mancheranno i nostri caffè, i nostri gossip sugli uomini, le nostre avventure e disavventure soprattutto, le sigarette in veranda con la coperta e le videochiamate. Mi mancheranno i nostri pranzi sulla panchina, le corse per prendere il pulmann, ma credo che il concerto di Gaia non ti mancherà affatto! 
Mi mancherà il tuo cappotto cammello orribile, il tuo modo di gesticolare quando la situazione si sta facendo interessante, il tuo modo di essere ironico che va in contrasto con il mio modo di fare ironia. 
Sarebbe scontato dirti che ti voglio bene, ma grazie perché senza di te questo percorso non sarebbe stato lo stesso.
\@

Mader: Sapevo dal primo momento che non ci saremmo sopportati, ma non sarei stata mai cosciente del fatto che sarebbe nata un’amicizia così importante come la nostra. 
Ho sempre creduto che tu avessi quel qualcosa in più che poteva contraddistinguere questo lungo percorso, e così è stato. 
Abbiamo fatto tanta strada insieme, mille viaggi, mille canzoni stonate cantate, mille incazzature, birre e spritz. Abbiamo condiviso palchi, abbiamo condiviso un disco, abbiamo condiviso concerti, abbiamo condiviso molti sorrisi. 
Mader non sarà mai scontato per me, ma Pietro sarà sempre un amico su cui potrò contare, anche oltre il conservatorio, so che nella mia vita avrai un posticino importante su cui ci sarà scritto sempre il tuo nome. 
Ricordati Mader senza Perla, non sarà mai lo stesso. 
Ti voglio bene. 
\@

Ringrazio il gruppo Avanzato, grazie perché in questi anni, avete portato quella boccata di aria fresca che mi è servita per rendere tutto questo più leggero. La danza ha fatto sempre parte della mia vita, e voi ne siete l’esempio. 
La nostra amicizia avrà sempre spazio nel mio cuore, e ognuno di voi ha reso tutto questo incancellabile. 
\@

Ringrazio tutto il mio gruppo di amici: 
Antonella, Arianna, Azzurra, Eleonora, Ciccio, Giorgia, Niccolò, Pier, Pietro, Benny, Paride, Raffa, Gabri, Serena, Roberto, Tim, Davide, Simona, Vanessa, grazie perché in questi due anni la mia salvezza siete stati voi. 
So che non è stato semplice all’inizio integrarmi in questo magico gruppo, ma sono contenta che ad oggi, ognuno di voi ha dentro di me un posto nel mio cuore. 
Grazie per le serate, per i calici, per l’obbligo o verità, per le vostre battute sempre fuori luogo, che mi hanno fatto sempre ridere. Grazie per essere venuti ai miei concerti, grazie perché siete i miei primi fan, nel bene o nel male, mi avete sempre sostenuta. 
\@

In particolare volevo ringraziare
Giulia: Sei l’unica che in questo percorso è entrata per davvero, un’entrata un po' scombussolante, ma che ti è rimasta dentro. 
Nonostante tutto, in questi anni, la nostra amicizia ha preso un percorso che non mi sarei mai immaginata. Ci siamo supportate a vicenda, e tu hai sempre cercato di interessarti a questo mondo, quando ne parlavamo, tra le mille cose. 
In punta di piedi ci siamo legate entrambe, una all’altra, e nonostante le mille divergenze dei nostri modi di pensare e di essere, ci siamo sempre trovate. 
Grazie per la tua gentilezza, la tua bontà e pacatezza, nell’avermi aiutata quando ho avuto i miei momenti no, e quando nei momenti si, ci siamo divertite tantissimo. 
Grazie perché so che potrò contare sempre su di te. 
\@

Volevo concludere questo capitolo ringraziando la mia famiglia. 
\@

Mamma, Papà, Emilio, grazie perché siete il mio punto di forza che non cadrà mai. 
La famiglia è sempre il luogo in cui ho potuto essere me stessa, soprattutto con la musica. 
Siete i miei primi fan, i miei primi sostenitori, e non avete mai mollato un solo giorno. Ci avete sempre creduto quasi quanto me. 
Grazie per la pazienza di questi anni, eravate consapevoli che questo percorso non sarebbe mai stato semplice, ma so che siete comunque nonostante tutto, fieri di me, e spero di avervi regalato un briciolo di felicità per i miei traguardi. 
Vi voglio bene, ma questo lo sapete già. 
\@

Infine ringrazio me stessa, l’amore per la musica, il non aver mai mollato in questo percorso, nonostante mi potessi sentire sempre messa in un angolo. Quell’angolo mi ha aiutata a capire quanto per me questa strada sia importante. 
La mia forza, la mia testardaggine, il mio essere così inaspettata, mi ha portato alla consapevolezza, che la mia voce, la mia musica hanno un valore, e io lo riconosco ogni giorno appena sento il bisogno di ascoltarla. 
Ho imparato che la musica non ha confine, questo percorso è stato fondamentale, perché non mi sarei mai immaginata di sceglierlo e portarlo avanti. 
\@

Grazie, \textit{“Ad Maiora Semper”}
